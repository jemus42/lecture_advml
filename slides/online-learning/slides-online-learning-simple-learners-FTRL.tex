\documentclass[11pt,compress,t,notes=noshow, xcolor=table]{beamer}
\usepackage[]{graphicx}
\usepackage[]{color}
% maxwidth is the original width if it is less than linewidth
% otherwise use linewidth (to make sure the graphics do not exceed the margin)
\makeatletter
\def\maxwidth{ %
  \ifdim\Gin@nat@width>\linewidth
    \linewidth
  \else
    \Gin@nat@width
  \fi
}
\makeatother

% ---------------------------------%
% latex-math dependencies, do not remove:
% - \usepackage{mathtools}
% - \usepackage{bm}
% - \usepackage{siunitx}
% - \usepackage{dsfont}
% - \usepackage{xspace}
% ---------------------------------%

%--------------------------------------------------------%
%       Language, encoding, typography
%--------------------------------------------------------%

\usepackage[english]{babel}
\usepackage[utf8]{inputenc} % Enables inputting UTF-8 symbols
% Standard AMS suite
\usepackage{amsmath,amsfonts,amssymb}

% Font four double-stroke / blackboard letters for sets of numbers (N, R, ...)
% Distribution name is "doublestroke"
% According to https://mirror.physik.tu-berlin.de/pub/CTAN/fonts/doublestroke/dsdoc.pdf
% the "bbm" package does a similar thing and may be superfluous.
% Required for latex-math
\usepackage{dsfont}

% bbm – "Blackboard-style" cm fonts (https://www.ctan.org/pkg/bbm)
% Used to be in common.tex, loaded directly after this file
% Maybe superfluous given dsfont is loaded
% TODO: Check if really unused?
% \usepackage{bbm}

% bm – Access bold symbols in maths mode - https://ctan.org/pkg/bm
% Required for latex-math
% https://tex.stackexchange.com/questions/3238/bm-package-versus-boldsymbol
\usepackage{bm}

% pifont – Access to PostScript standard Symbol and Dingbats fonts
% Used for \newcommand{\xmark}{\ding{55}, which is never used
% aside from lecture_advml/attic/xx-automl/slides.Rnw
% \usepackage{pifont}

% Quotes (inline and display), provdes \enquote
% https://ctan.org/pkg/csquotes
\usepackage{csquotes}

% Adds arg to enumerate env, technically superseded by enumitem according
% to https://ctan.org/pkg/enumerate
% Replace with https://ctan.org/pkg/enumitem ?
\usepackage{enumerate}

% Line spacing - provides \singlespacing \doublespacing \onehalfspacing
% https://ctan.org/pkg/setspace
% TODO: Check if really unused?
%\usepackage{setspace}

% mathtools – Mathematical tools to use with amsmath
% https://ctan.org/pkg/mathtools?lang=en
% latex-math dependency according to latex-math repo
\usepackage{mathtools}

%--------------------------------------------------------%
%       Displaying code and algorithms
%--------------------------------------------------------%
\usepackage{verbatim}
\usepackage{algorithm}
\usepackage{algpseudocode}

%--------------------------------------------------------%
%       Tables
%--------------------------------------------------------%

% multi-row table cells: https://www.namsu.de/Extra/pakete/Multirow.html
\usepackage{multirow}

% long/multi-page tables: https://texdoc.org/serve/longtable.pdf/0
% TODO: Check if really unused?

\usepackage{longtable}

% pretty table env: https://ctan.org/pkg/booktabs?lang=en
% TODO: Check if really unused?
\usepackage{booktabs}

%--------------------------------------------------------%
%       Figures: Creating, placing, verbing
%--------------------------------------------------------%

% wrapfig - Wrapping text around figures https://de.overleaf.com/learn/latex/Wrapping_text_around_figures
\usepackage{wrapfig}

% Sub figures in figures and tables
% https://ctan.org/pkg/subfig -- supersedes subfigure package
% TODO: Check if really unused?
\usepackage{subfig}

% Actually it's pronounced PGF https://en.wikibooks.org/wiki/LaTeX/PGF/TikZ
\usepackage{tikz}

\usetikzlibrary{shapes,arrows,automata,positioning,calc,chains,trees, shadows}
\tikzset{
  %Define standard arrow tip
  >=stealth',
  %Define style for boxes
  punkt/.style={
    rectangle,
    rounded corners,
    draw=black, very thick,
    text width=6.5em,
    minimum height=2em,
    text centered},
  % Define arrow style
  pil/.style={
    ->,
    thick,
    shorten <=2pt,
    shorten >=2pt,}
}


% Unsorted
% textpos – Place boxes at arbitrary positions on the LATEX page
% https://ctan.org/pkg/textpos?lang=en
% Provides \begin{textblock}
 % TODO: Check if really unused?
\usepackage[absolute,overlay]{textpos}

% psfrag – Replace strings in encapsulated PostScript figures
% https://www.overleaf.com/latex/examples/psfrag-example/tggxhgzwrzhn
% https://ftp.mpi-inf.mpg.de/pub/tex/mirror/ftp.dante.de/pub/tex/macros/latex/contrib/psfrag/pfgguide.pdf
% Can't tell if this is needed
% TODO: Check if really unused?
\usepackage{psfrag}

% Maybe not great to use this https://tex.stackexchange.com/a/197/19093
% Use align instead -- TODO: Global search & replace to check
\usepackage{eqnarray}

\usepackage{colortbl}

% arydshln – Draw dash-lines in array/tabular
% https://www.ctan.org/pkg/arydshln
% !! "arydshln has to be loaded after array, longtable, colortab and/or colortbl"
% Provides \hdashline and \cdashline
% TODO: Check if really unused?
% \usepackage{arydshln}

% tabularx – Tabulars with adjustable-width columns
% https://ctan.org/pkg/tabularx
% Provides \begin{tabularx}
% TODO: Check if really unused?
% \usepackage{tabularx}

% placeins – Control float placement
% https://ctan.org/pkg/placeins
% Defines a \FloatBarrier command
% TODO: Check if really unused?
% \usepackage{placeins}


% framed – Framed or shaded regions that can break across pages
% https://ctan.org/pkg/framed
% Provides \begin{framed} which uses \colorbox{shadecolor} relying on \definecolor{shadecolor}.
% TODO: Check if really unused?
% \usepackage{framed}

% Used often in conjunction with \definecolor{shadecolor}{rgb}{0.969, 0.969, 0.969}
% Might be able to be removed or at least redefined to only have shadecolor (if needed)
\definecolor{fgcolor}{rgb}{0.345, 0.345, 0.345}
\definecolor{shadecolor}{rgb}{0.969, 0.969, 0.969}
\newenvironment{knitrout}{}{} % an empty environment to be redefined in TeX


% Defines macros and environments
\usepackage{../../style/lmu-lecture}

\let\code=\texttt % Used regularly
\let\proglang=\textsf % Unused?

% Not sure what/why this does
\setkeys{Gin}{width=0.9\textwidth}

\setbeamertemplate{frametitle}{\expandafter\uppercase\expandafter\insertframetitle}

% Can't find a reason why common.tex is not just part of this file?
% Rarely used fontstyle for R packages, used only in 
% - forests/slides-forests-benchmark.tex
% - exercises/single-exercises/methods_l_1.Rnw
% - slides/cart/attic/slides_extra_trees.Rnw
\newcommand{\pkg}[1]{{\fontseries{b}\selectfont #1}}

% Spacing helpers, used often (mostly in exercises for \dlz)
\newcommand{\lz}{\vspace{0.5cm}} % vertical space (used often in slides)
\newcommand{\dlz}{\vspace{1cm}}  % double vertical space (used often in exercises, never in slides)

%--------------------%
%  New environments  %
%--------------------%

 % Frame with breaks and verbatim // this is used very often
\newenvironment{vbframe}
{
\begin{frame}[containsverbatim,allowframebreaks]
}
{
\end{frame}
}

% Frame with verbatim without breaks (to avoid numbering one slided frames)
% This is not used anywhere but I can see it being useful
\newenvironment{vframe}
{
\begin{frame}[containsverbatim]
}
{
\end{frame}
}

% Itemize block
\newenvironment{blocki}[1]
{
\begin{block}{#1}\begin{itemize}
}
{
\end{itemize}\end{block}
}

% textcolor that works in mathmode
% https://tex.stackexchange.com/a/261480
% Used e.g. in forests/slides-forests-bagging.tex
% [...] \textcolor{blue}{\tfrac{1}{M}\sum^M_{m} [...]
% \makeatletter
% \renewcommand*{\@textcolor}[3]{%
%   \protect\leavevmode
%   \begingroup
%     \color#1{#2}#3%
%   \endgroup
% }
% \makeatother


%-------------------------------------------------------------------------------------------------------%
%  Unused stuff that needs to go but is kept here currently juuuust in case it was important after all  %
%-------------------------------------------------------------------------------------------------------%

% \newcommand{\hlnum}[1]{\textcolor[rgb]{0.686,0.059,0.569}{#1}}%
% \newcommand{\hlstr}[1]{\textcolor[rgb]{0.192,0.494,0.8}{#1}}%
% \newcommand{\hlcom}[1]{\textcolor[rgb]{0.678,0.584,0.686}{\textit{#1}}}%
% \newcommand{\hlopt}[1]{\textcolor[rgb]{0,0,0}{#1}}%
% \newcommand{\hlstd}[1]{\textcolor[rgb]{0.345,0.345,0.345}{#1}}%
% \newcommand{\hlkwa}[1]{\textcolor[rgb]{0.161,0.373,0.58}{\textbf{#1}}}%
% \newcommand{\hlkwb}[1]{\textcolor[rgb]{0.69,0.353,0.396}{#1}}%
% \newcommand{\hlkwc}[1]{\textcolor[rgb]{0.333,0.667,0.333}{#1}}%
% \newcommand{\hlkwd}[1]{\textcolor[rgb]{0.737,0.353,0.396}{\textbf{#1}}}%
% \let\hlipl\hlkwb

% \makeatletter
% \newenvironment{kframe}{%
%  \def\at@end@of@kframe{}%
%  \ifinner\ifhmode%
%   \def\at@end@of@kframe{\end{minipage}}%
%   \begin{minipage}{\columnwidth}%
%  \fi\fi%
%  \def\FrameCommand##1{\hskip\@totalleftmargin \hskip-\fboxsep
%  \colorbox{shadecolor}{##1}\hskip-\fboxsep
%      % There is no \\@totalrightmargin, so:
%      \hskip-\linewidth \hskip-\@totalleftmargin \hskip\columnwidth}%
%  \MakeFramed {\advance\hsize-\width
%    \@totalleftmargin\z@ \linewidth\hsize
%    \@setminipage}}%
%  {\par\unskip\endMakeFramed%
%  \at@end@of@kframe}
% \makeatother

% \definecolor{shadecolor}{rgb}{.97, .97, .97}
% \definecolor{messagecolor}{rgb}{0, 0, 0}
% \definecolor{warningcolor}{rgb}{1, 0, 1}
% \definecolor{errorcolor}{rgb}{1, 0, 0}
% \newenvironment{knitrout}{}{} % an empty environment to be redefined in TeX

% \usepackage{alltt}
% \newcommand{\SweaveOpts}[1]{}  % do not interfere with LaTeX
% \newcommand{\SweaveInput}[1]{} % because they are not real TeX commands
% \newcommand{\Sexpr}[1]{}       % will only be parsed by R
% \newcommand{\xmark}{\ding{55}}%

% dependencies: amsmath, amssymb, dsfont
% math spaces
\ifdefined\N
\renewcommand{\N}{\mathds{N}} % N, naturals
\else \newcommand{\N}{\mathds{N}} \fi
\newcommand{\Z}{\mathds{Z}} % Z, integers
\newcommand{\Q}{\mathds{Q}} % Q, rationals
\newcommand{\R}{\mathds{R}} % R, reals
\ifdefined\C
\renewcommand{\C}{\mathds{C}} % C, complex
\else \newcommand{\C}{\mathds{C}} \fi
\newcommand{\continuous}{\mathcal{C}} % C, space of continuous functions
\newcommand{\M}{\mathcal{M}} % machine numbers
\newcommand{\epsm}{\epsilon_m} % maximum error

% counting / finite sets
\newcommand{\setzo}{\{0, 1\}} % set 0, 1
\newcommand{\setmp}{\{-1, +1\}} % set -1, 1
\newcommand{\unitint}{[0, 1]} % unit interval

% basic math stuff
\newcommand{\xt}{\tilde x} % x tilde
\DeclareMathOperator*{\argmax}{arg\,max} % argmax
\DeclareMathOperator*{\argmin}{arg\,min} % argmin
\newcommand{\argminlim}{\mathop{\mathrm{arg\,min}}\limits} % argmax with limits
\newcommand{\argmaxlim}{\mathop{\mathrm{arg\,max}}\limits} % argmin with limits
\newcommand{\sign}{\operatorname{sign}} % sign, signum
\newcommand{\I}{\mathbb{I}} % I, indicator
\newcommand{\order}{\mathcal{O}} % O, order
\newcommand{\bigO}{\mathcal{O}} % Big-O Landau
\newcommand{\littleo}{{o}} % Little-o Landau
\newcommand{\pd}[2]{\frac{\partial{#1}}{\partial #2}} % partial derivative
\newcommand{\floorlr}[1]{\left\lfloor #1 \right\rfloor} % floor
\newcommand{\ceillr}[1]{\left\lceil #1 \right\rceil} % ceiling
\newcommand{\indep}{\perp \!\!\! \perp} % independence symbol

% sums and products
\newcommand{\sumin}{\sum\limits_{i=1}^n} % summation from i=1 to n
\newcommand{\sumim}{\sum\limits_{i=1}^m} % summation from i=1 to m
\newcommand{\sumjn}{\sum\limits_{j=1}^n} % summation from j=1 to p
\newcommand{\sumjp}{\sum\limits_{j=1}^p} % summation from j=1 to p
\newcommand{\sumik}{\sum\limits_{i=1}^k} % summation from i=1 to k
\newcommand{\sumkg}{\sum\limits_{k=1}^g} % summation from k=1 to g
\newcommand{\sumjg}{\sum\limits_{j=1}^g} % summation from j=1 to g
\newcommand{\meanin}{\frac{1}{n} \sum\limits_{i=1}^n} % mean from i=1 to n
\newcommand{\meanim}{\frac{1}{m} \sum\limits_{i=1}^m} % mean from i=1 to n
\newcommand{\meankg}{\frac{1}{g} \sum\limits_{k=1}^g} % mean from k=1 to g
\newcommand{\prodin}{\prod\limits_{i=1}^n} % product from i=1 to n
\newcommand{\prodkg}{\prod\limits_{k=1}^g} % product from k=1 to g
\newcommand{\prodjp}{\prod\limits_{j=1}^p} % product from j=1 to p

% linear algebra
\newcommand{\one}{\bm{1}} % 1, unitvector
\newcommand{\zero}{\mathbf{0}} % 0-vector
\newcommand{\id}{\bm{I}} % I, identity
\newcommand{\diag}{\operatorname{diag}} % diag, diagonal
\newcommand{\trace}{\operatorname{tr}} % tr, trace
\newcommand{\spn}{\operatorname{span}} % span
\newcommand{\scp}[2]{\left\langle #1, #2 \right\rangle} % <.,.>, scalarproduct
\newcommand{\mat}[1]{\begin{pmatrix} #1 \end{pmatrix}} % short pmatrix command
\newcommand{\Amat}{\mathbf{A}} % matrix A
\newcommand{\Deltab}{\mathbf{\Delta}} % error term for vectors

% basic probability + stats
\renewcommand{\P}{\mathds{P}} % P, probability
\newcommand{\E}{\mathds{E}} % E, expectation
\newcommand{\var}{\mathsf{Var}} % Var, variance
\newcommand{\cov}{\mathsf{Cov}} % Cov, covariance
\newcommand{\corr}{\mathsf{Corr}} % Corr, correlation
\newcommand{\normal}{\mathcal{N}} % N of the normal distribution
\newcommand{\iid}{\overset{i.i.d}{\sim}} % dist with i.i.d superscript
\newcommand{\distas}[1]{\overset{#1}{\sim}} % ... is distributed as ...

% machine learning
\newcommand{\Xspace}{\mathcal{X}} % X, input space
\newcommand{\Yspace}{\mathcal{Y}} % Y, output space
\newcommand{\Zspace}{\mathcal{Z}} % Space of sampled datapoints ! Also defined identically in ml-online.tex !
\newcommand{\nset}{\{1, \ldots, n\}} % set from 1 to n
\newcommand{\pset}{\{1, \ldots, p\}} % set from 1 to p
\newcommand{\gset}{\{1, \ldots, g\}} % set from 1 to g
\newcommand{\Pxy}{\mathbb{P}_{xy}} % P_xy
\newcommand{\Exy}{\mathbb{E}_{xy}} % E_xy: Expectation over random variables xy
\newcommand{\xv}{\mathbf{x}} % vector x (bold)
\newcommand{\xtil}{\tilde{\mathbf{x}}} % vector x-tilde (bold)
\newcommand{\yv}{\mathbf{y}} % vector y (bold)
\newcommand{\xy}{(\xv, y)} % observation (x, y)
\newcommand{\xvec}{\left(x_1, \ldots, x_p\right)^\top} % (x1, ..., xp)
\newcommand{\Xmat}{\mathbf{X}} % Design matrix
\newcommand{\allDatasets}{\mathds{D}} % The set of all datasets
\newcommand{\allDatasetsn}{\mathds{D}_n}  % The set of all datasets of size n
\newcommand{\D}{\mathcal{D}} % D, data
\newcommand{\Dn}{\D_n} % D_n, data of size n
\newcommand{\Dtrain}{\mathcal{D}_{\text{train}}} % D_train, training set
\newcommand{\Dtest}{\mathcal{D}_{\text{test}}} % D_test, test set
\newcommand{\xyi}[1][i]{\left(\xv^{(#1)}, y^{(#1)}\right)} % (x^i, y^i), i-th observation
\newcommand{\Dset}{\left( \xyi[1], \ldots, \xyi[n]\right)} % {(x1,y1)), ..., (xn,yn)}, data
\newcommand{\defAllDatasetsn}{(\Xspace \times \Yspace)^n} % Def. of the set of all datasets of size n
\newcommand{\defAllDatasets}{\bigcup_{n \in \N}(\Xspace \times \Yspace)^n} % Def. of the set of all datasets
\newcommand{\xdat}{\left\{ \xv^{(1)}, \ldots, \xv^{(n)}\right\}} % {x1, ..., xn}, input data
\newcommand{\ydat}{\left\{ \yv^{(1)}, \ldots, \yv^{(n)}\right\}} % {y1, ..., yn}, input data
\newcommand{\yvec}{\left(y^{(1)}, \hdots, y^{(n)}\right)^\top} % (y1, ..., yn), vector of outcomes
\renewcommand{\xi}[1][i]{\xv^{(#1)}} % x^i, i-th observed value of x
\newcommand{\yi}[1][i]{y^{(#1)}} % y^i, i-th observed value of y
\newcommand{\xivec}{\left(x^{(i)}_1, \ldots, x^{(i)}_p\right)^\top} % (x1^i, ..., xp^i), i-th observation vector
\newcommand{\xj}{\xv_j} % x_j, j-th feature
\newcommand{\xjvec}{\left(x^{(1)}_j, \ldots, x^{(n)}_j\right)^\top} % (x^1_j, ..., x^n_j), j-th feature vector
\newcommand{\phiv}{\mathbf{\phi}} % Basis transformation function phi
\newcommand{\phixi}{\mathbf{\phi}^{(i)}} % Basis transformation of xi: phi^i := phi(xi)

%%%%%% ml - models general
\newcommand{\lamv}{\bm{\lambda}} % lambda vector, hyperconfiguration vector
\newcommand{\Lam}{\bm{\Lambda}}	 % Lambda, space of all hpos
% Inducer / Inducing algorithm
\newcommand{\preimageInducer}{\left(\defAllDatasets\right)\times\Lam} % Set of all datasets times the hyperparameter space
\newcommand{\preimageInducerShort}{\allDatasets\times\Lam} % Set of all datasets times the hyperparameter space
% Inducer / Inducing algorithm
\newcommand{\ind}{\mathcal{I}} % Inducer, inducing algorithm, learning algorithm

% continuous prediction function f
\newcommand{\ftrue}{f_{\text{true}}}  % True underlying function (if a statistical model is assumed)
\newcommand{\ftruex}{\ftrue(\xv)} % True underlying function (if a statistical model is assumed)
\newcommand{\fx}{f(\xv)} % f(x), continuous prediction function
\newcommand{\fdomains}{f: \Xspace \rightarrow \R^g} % f with domain and co-domain
\newcommand{\Hspace}{\mathcal{H}} % hypothesis space where f is from
\newcommand{\fbayes}{f^{\ast}} % Bayes-optimal model
\newcommand{\fxbayes}{f^{\ast}(\xv)} % Bayes-optimal model
\newcommand{\fkx}[1][k]{f_{#1}(\xv)} % f_j(x), discriminant component function
\newcommand{\fh}{\hat{f}} % f hat, estimated prediction function
\newcommand{\fxh}{\fh(\xv)} % fhat(x)
\newcommand{\fxt}{f(\xv ~|~ \thetab)} % f(x | theta)
\newcommand{\fxi}{f\left(\xv^{(i)}\right)} % f(x^(i))
\newcommand{\fxih}{\hat{f}\left(\xv^{(i)}\right)} % f(x^(i))
\newcommand{\fxit}{f\left(\xv^{(i)} ~|~ \thetab\right)} % f(x^(i) | theta)
\newcommand{\fhD}{\fh_{\D}} % fhat_D, estimate of f based on D
\newcommand{\fhDtrain}{\fh_{\Dtrain}} % fhat_Dtrain, estimate of f based on D
\newcommand{\fhDnlam}{\fh_{\Dn, \lamv}} %model learned on Dn with hp lambda
\newcommand{\fhDlam}{\fh_{\D, \lamv}} %model learned on D with hp lambda
\newcommand{\fhDnlams}{\fh_{\Dn, \lamv^\ast}} %model learned on Dn with optimal hp lambda
\newcommand{\fhDlams}{\fh_{\D, \lamv^\ast}} %model learned on D with optimal hp lambda

% discrete prediction function h
\newcommand{\hx}{h(\xv)} % h(x), discrete prediction function
\newcommand{\hh}{\hat{h}} % h hat
\newcommand{\hxh}{\hat{h}(\xv)} % hhat(x)
\newcommand{\hxt}{h(\xv | \thetab)} % h(x | theta)
\newcommand{\hxi}{h\left(\xi\right)} % h(x^(i))
\newcommand{\hxit}{h\left(\xi ~|~ \thetab\right)} % h(x^(i) | theta)
\newcommand{\hbayes}{h^{\ast}} % Bayes-optimal classification model
\newcommand{\hxbayes}{h^{\ast}(\xv)} % Bayes-optimal classification model

% yhat
\newcommand{\yh}{\hat{y}} % yhat for prediction of target
\newcommand{\yih}{\hat{y}^{(i)}} % yhat^(i) for prediction of ith targiet
\newcommand{\resi}{\yi- \yih}

% theta
\newcommand{\thetah}{\hat{\theta}} % theta hat
\newcommand{\thetab}{\bm{\theta}} % theta vector
\newcommand{\thetabh}{\bm{\hat\theta}} % theta vector hat
\newcommand{\thetat}[1][t]{\thetab^{[#1]}} % theta^[t] in optimization
\newcommand{\thetatn}[1][t]{\thetab^{[#1 +1]}} % theta^[t+1] in optimization
\newcommand{\thetahDnlam}{\thetabh_{\Dn, \lamv}} %theta learned on Dn with hp lambda
\newcommand{\thetahDlam}{\thetabh_{\D, \lamv}} %theta learned on D with hp lambda
\newcommand{\mint}{\min_{\thetab \in \Theta}} % min problem theta
\newcommand{\argmint}{\argmin_{\thetab \in \Theta}} % argmin theta

% densities + probabilities
% pdf of x
\newcommand{\pdf}{p} % p
\newcommand{\pdfx}{p(\xv)} % p(x)
\newcommand{\pixt}{\pi(\xv~|~ \thetab)} % pi(x|theta), pdf of x given theta
\newcommand{\pixit}[1][i]{\pi\left(\xi[#1] ~|~ \thetab\right)} % pi(x^i|theta), pdf of x given theta
\newcommand{\pixii}[1][i]{\pi\left(\xi[#1]\right)} % pi(x^i), pdf of i-th x

% pdf of (x, y)
\newcommand{\pdfxy}{p(\xv,y)} % p(x, y)
\newcommand{\pdfxyt}{p(\xv, y ~|~ \thetab)} % p(x, y | theta)
\newcommand{\pdfxyit}{p\left(\xi, \yi ~|~ \thetab\right)} % p(x^(i), y^(i) | theta)

% pdf of x given y
\newcommand{\pdfxyk}[1][k]{p(\xv | y= #1)} % p(x | y = k)
\newcommand{\lpdfxyk}[1][k]{\log p(\xv | y= #1)} % log p(x | y = k)
\newcommand{\pdfxiyk}[1][k]{p\left(\xi | y= #1 \right)} % p(x^i | y = k)

% prior probabilities
\newcommand{\pik}[1][k]{\pi_{#1}} % pi_k, prior
\newcommand{\lpik}[1][k]{\log \pi_{#1}} % log pi_k, log of the prior
\newcommand{\pit}{\pi(\thetab)} % Prior probability of parameter theta

% posterior probabilities
\newcommand{\post}{\P(y = 1 ~|~ \xv)} % P(y = 1 | x), post. prob for y=1
\newcommand{\postk}[1][k]{\P(y = #1 ~|~ \xv)} % P(y = k | y), post. prob for y=k
\newcommand{\pidomains}{\pi: \Xspace \rightarrow \unitint} % pi with domain and co-domain
\newcommand{\pibayes}{\pi^{\ast}} % Bayes-optimal classification model
\newcommand{\pixbayes}{\pi^{\ast}(\xv)} % Bayes-optimal classification model
\newcommand{\pix}{\pi(\xv)} % pi(x), P(y = 1 | x)
\newcommand{\piv}{\bm{\pi}} % pi, bold, as vector
\newcommand{\pikx}[1][k]{\pi_{#1}(\xv)} % pi_k(x), P(y = k | x)
\newcommand{\pikxt}[1][k]{\pi_{#1}(\xv ~|~ \thetab)} % pi_k(x | theta), P(y = k | x, theta)
\newcommand{\pixh}{\hat \pi(\xv)} % pi(x) hat, P(y = 1 | x) hat
\newcommand{\pikxh}[1][k]{\hat \pi_{#1}(\xv)} % pi_k(x) hat, P(y = k | x) hat
\newcommand{\pixih}{\hat \pi(\xi)} % pi(x^(i)) with hat
\newcommand{\pikxih}[1][k]{\hat \pi_{#1}(\xi)} % pi_k(x^(i)) with hat
\newcommand{\pdfygxt}{p(y ~|~\xv, \thetab)} % p(y | x, theta)
\newcommand{\pdfyigxit}{p\left(\yi ~|~\xi, \thetab\right)} % p(y^i |x^i, theta)
\newcommand{\lpdfygxt}{\log \pdfygxt } % log p(y | x, theta)
\newcommand{\lpdfyigxit}{\log \pdfyigxit} % log p(y^i |x^i, theta)

% probababilistic
\newcommand{\bayesrulek}[1][k]{\frac{\P(\xv | y= #1) \P(y= #1)}{\P(\xv)}} % Bayes rule
\newcommand{\muk}{\bm{\mu_k}} % mean vector of class-k Gaussian (discr analysis)

% residual and margin
\newcommand{\eps}{\epsilon} % residual, stochastic
\newcommand{\epsi}{\epsilon^{(i)}} % epsilon^i, residual, stochastic
\newcommand{\epsh}{\hat{\epsilon}} % residual, estimated
\newcommand{\yf}{y \fx} % y f(x), margin
\newcommand{\yfi}{\yi \fxi} % y^i f(x^i), margin
\newcommand{\Sigmah}{\hat \Sigma} % estimated covariance matrix
\newcommand{\Sigmahj}{\hat \Sigma_j} % estimated covariance matrix for the j-th class

% ml - loss, risk, likelihood
\newcommand{\Lyf}{L\left(y, f\right)} % L(y, f), loss function
\newcommand{\Lypi}{L\left(y, \pi\right)} % L(y, pi), loss function
\newcommand{\Lxy}{L\left(y, \fx\right)} % L(y, f(x)), loss function
\newcommand{\Lxyi}{L\left(\yi, \fxi\right)} % loss of observation
\newcommand{\Lxyt}{L\left(y, \fxt\right)} % loss with f parameterized
\newcommand{\Lxyit}{L\left(\yi, \fxit\right)} % loss of observation with f parameterized
\newcommand{\Lxym}{L\left(\yi, f\left(\bm{\tilde{x}}^{(i)} ~|~ \thetab\right)\right)} % loss of observation with f parameterized
\newcommand{\Lpixy}{L\left(y, \pix\right)} % loss in classification
\newcommand{\Lpiv}{L\left(y, \piv\right)} % loss in classification
\newcommand{\Lpixyi}{L\left(\yi, \pixii\right)} % loss of observation in classification
\newcommand{\Lpixyt}{L\left(y, \pixt\right)} % loss with pi parameterized
\newcommand{\Lpixyit}{L\left(\yi, \pixit\right)} % loss of observation with pi parameterized
\newcommand{\Lhxy}{L\left(y, \hx\right)} % L(y, h(x)), loss function on discrete classes
\newcommand{\Lr}{L\left(r\right)} % L(r), loss defined on residual (reg) / margin (classif)
\newcommand{\lone}{|y - \fx|} % L1 loss
\newcommand{\ltwo}{\left(y - \fx\right)^2} % L2 loss
\newcommand{\lbernoullimp}{\ln(1 + \exp(-y \cdot \fx))} % Bernoulli loss for -1, +1 encoding
\newcommand{\lbernoullizo}{- y \cdot \fx + \log(1 + \exp(\fx))} % Bernoulli loss for 0, 1 encoding
\newcommand{\lcrossent}{- y \log \left(\pix\right) - (1 - y) \log \left(1 - \pix\right)} % cross-entropy loss
\newcommand{\lbrier}{\left(\pix - y \right)^2} % Brier score
\newcommand{\risk}{\mathcal{R}} % R, risk
\newcommand{\riskbayes}{\mathcal{R}^\ast}
\newcommand{\riskf}{\risk(f)} % R(f), risk
\newcommand{\riskdef}{\E_{y|\xv}\left(\Lxy \right)} % risk def (expected loss)
\newcommand{\riskt}{\mathcal{R}(\thetab)} % R(theta), risk
\newcommand{\riske}{\mathcal{R}_{\text{emp}}} % R_emp, empirical risk w/o factor 1 / n
\newcommand{\riskeb}{\bar{\mathcal{R}}_{\text{emp}}} % R_emp, empirical risk w/ factor 1 / n
\newcommand{\riskef}{\riske(f)} % R_emp(f)
\newcommand{\risket}{\mathcal{R}_{\text{emp}}(\thetab)} % R_emp(theta)
\newcommand{\riskr}{\mathcal{R}_{\text{reg}}} % R_reg, regularized risk
\newcommand{\riskrt}{\mathcal{R}_{\text{reg}}(\thetab)} % R_reg(theta)
\newcommand{\riskrf}{\riskr(f)} % R_reg(f)
\newcommand{\riskrth}{\hat{\mathcal{R}}_{\text{reg}}(\thetab)} % hat R_reg(theta)
\newcommand{\risketh}{\hat{\mathcal{R}}_{\text{emp}}(\thetab)} % hat R_emp(theta)
\newcommand{\LL}{\mathcal{L}} % L, likelihood
\newcommand{\LLt}{\mathcal{L}(\thetab)} % L(theta), likelihood
\newcommand{\LLtx}{\mathcal{L}(\thetab | \xv)} % L(theta|x), likelihood
\newcommand{\logl}{\ell} % l, log-likelihood
\newcommand{\loglt}{\logl(\thetab)} % l(theta), log-likelihood
\newcommand{\logltx}{\logl(\thetab | \xv)} % l(theta|x), log-likelihood
\newcommand{\errtrain}{\text{err}_{\text{train}}} % training error
\newcommand{\errtest}{\text{err}_{\text{test}}} % test error
\newcommand{\errexp}{\overline{\text{err}_{\text{test}}}} % avg training error

% lm
\newcommand{\thx}{\thetab^\top \xv} % linear model
\newcommand{\olsest}{(\Xmat^\top \Xmat)^{-1} \Xmat^\top \yv} % OLS estimator in LM

\newcommand{\Aspace}{\mathcal{A}}
\newcommand{\norm}[1]{\left|\left|#1\right|\right|_2}
\newcommand{\llin}{L^{\texttt{lin}}}
\newcommand{\lzeroone}{L^{0-1}}
\newcommand{\lhinge}{L^{\texttt{hinge}}}
\newcommand{\lexphinge}{\widetilde{L^{\texttt{hinge}}}}
\newcommand{\lconv}{L^{\texttt{conv}}}
\newcommand{\FTL}{\texttt{FTL}}
\newcommand{\FTRL}{\texttt{FTRL}}
\newcommand{\OGD}{{\texttt{OGD}}}
\newcommand{\EWA}{{\texttt{EWA}}}
\newcommand{\REWA}{{\texttt{REWA}}}
\newcommand{\EXPthree}{{\texttt{EXP3}}}
\newcommand{\EXPthreep}{{\texttt{EXP3P}}}
\newcommand{\reg}{\psi}
\newcommand{\Algo}{\texttt{Algo}}


\usepackage{multicol}

\newcommand{\titlefigure}{figure/bias-variance-tradeoff}
\newcommand{\learninggoals}{ 
  \item Get to know FTRL as a stable alternative for FTL 
  \item See a suitable regularization for OLO problems
}

\title{Advanced Machine Learning}
\date{}

\begin{document}

\lecturechapter{Follow the regularized leader }
\lecture{Advanced Machine Learning}



\sloppy



\begin{frame} 
	\frametitle{Follow the regularized leader}
	% 	
	\small
	\begin{itemize}
		% 		
		\item To overcome the shortcomings of the FTL algorithm, one can incorporate a regularization function $\reg: \Aspace \to \R_+$ into the action choice of FTL, which leads to more stability.
		% 		
		 \item  
		To be more precise, let for $t\geq 1$
		%
		{\footnotesize 		\begin{align*} 
				%		\label{defi_forel_estimate}
				%	
				a_t^{\FTRL} \in \argmin_{a \in \mathcal{A}} \left( \reg(a) + \sum\nolimits_{s=1}^{t-1} \l(a,z_s) \right),
				%	
			\end{align*}
			{\tiny (Technical side note: if there are more than one minimum, then one of them is chosen.)\\}	
		}
		%
		\noindent then the algorithm choosing $a_t^{\FTRL}$ in time step $t$ is called the \textbf{Follow the regularized leader} (FTRL) algorithm.
		% 		
		  \item {\visible<2->{  \emph{Interpretation:} The algorithm predicts $a_t$ as the element in $\Aspace,$ which minimizes the regularization function plus the cumulative loss so far over the previous $t-1$ time periods.}}
		%		
		  \item {\visible<3>{  Obviously, the behavior of the FTRL algorithm is depending heavily on the choice of the regularization function $\reg.$ 
%		  
		  If $\reg \equiv 0,$ then FTRL equals FTL.}}
	\end{itemize}
	% 	
\end{frame}

\begin{frame} 	
	\frametitle{Regularization in online learning vs.\ batch learning}
	\small
	\begin{itemize}
		%	
		\item Note that in the batch learning scenario, the learner seeks to optimize an objective function which is the sum of the training loss and a regularization function:
		%
		\begin{align*}
			% 		 \label{def:form_of_ml_problems}
			%	
			\min_{\thetab\in \R^p} \, \sum_{i=1}^n L(\yi,\thetab) + \lambda \, \psi(\thetab),	
			%	
		\end{align*}	
		% 		
		where $\lambda\geq 0$ is some regularization parameter.
		%	
		{\visible<2->{  \item Here, the regularization function is part of the whole objective function, which the learner seeks to minimize. }}
		%	
		 \item {\visible<3>{  However, in the online learning scenario the regularization function does (usually) not appear in the regret the learner seeks to minimize, but the regularization function is only part of the action/decision rule at each time step.}}
		% 		
	\end{itemize}
	% 	
\end{frame}


\begin{frame} 
	\frametitle{Regret analysis of FTRL: A Helpful Lemma}
	%	
	\small
	\begin{itemize}
		\item \textbf{Lemma:}
		%	
		Let $a_1^{\FTRL}, a_2^{\FTRL}, \ldots$ be the sequence of actions coming used by the FTRL algorithm for the environmental data sequence $z_1,z_2,\ldots$ . 
		%	
		 Then, for all $\tilde a \in \Aspace$ we have
		%	
		\begin{equation*}
			%	
			\begin{split}
				R_T^{\FTRL}(\tilde a) &= \sum\limits_{t=1}^T \big(\l(a_t^{\FTRL},z_t) - \l(\tilde a,z_t) \big)
				%		
				\\ &\leq \reg(\tilde a) - \reg(a_1^{\FTRL}) +\sum\limits_{t=1}^T \left(\l(a_t^{\FTRL},z_t) - \l(a_{t+1}^{\FTRL},z_t)\right).
			\end{split}
			%	
		\end{equation*}
		%	
		\pause	
		  \item \emph{Interpretation}: the regret of the FTRL algorithm is bounded by the difference of cumulated losses of itself compared to its one-step lookahead cheater version and an additional regularization difference term. 
		%		
		  \item [$\Rightarrow$] We have seen an analogous result for FTL!
		  
		  {\tiny (The proof is similar.)}
		%	
	\end{itemize}
	% 	
\end{frame}


%\begin{frame} 
%	\frametitle{Regret analysis of FTRL: A Helpful Lemma}
%	%	
%	\small
%	\begin{itemize}
%		\footnotesize 
%		%	
%		\item \textbf{Proof:} 
%		\begin{itemize}
%			\item 		For sake of brevity, we write $a_1, a_2, \ldots$ for $a_1^{\FTRL}, a_2^{\FTRL}, \ldots$ 
%			
%			 \item Note that FTRL for $\l(\cdot,z_1), \ldots, \l(\cdot,z_T)$ is equivalent to running FTL on $\l(\cdot,z_0),\l(\cdot,z_1), \ldots, \l(\cdot,z_T)$ with $\l(\cdot,z_0)=\reg(\cdot),$ that is, we are pretending as if there was an additional time step $t=0,$ where we suffered a loss of  $\reg(\cdot)$ and performed a meaningless action $a_0.$
%			%		
%			  \item Indeed:
%			%		
%			$$	a_t^{\FTRL} = \argmin{a \in \mathcal{A}} \big( \reg(a) + \sum_{i=1}^{t-1} \l(a,z_i) \big) 
%			  = \argmin{a \in \mathcal{A}} \big( \sum_{i=0}^{t-1} \l(a,z_i) \big)   = a_t^{\FTL}.
%			%			
%			$$
%			%		
%			%		where the action for FTL is 
%			%	
%		\end{itemize}
%	\end{itemize}
%	% 	
%\end{frame}
%
%
%\begin{frame} 
%	\frametitle{Regret analysis of FTRL: A Helpful Lemma}
%	%	
%	\small
%	\begin{itemize}\item[]
%		\begin{itemize}
%			%		
%			\footnotesize
%			\item Thus, by applying the analogous lemma for the FTL, we obtain for any $\tilde a \in \Aspace$
%			%	
%			\begin{center}
%				%
%				$R_T^{FTL}(\tilde a) = \sum\nolimits_{t=0}^T (\l(a_t,z_t) - \l(\tilde a,z_t)) \leq  \sum\nolimits_{t=0}^T (\l(a_t,z_t) - \l(a_{t+1},z_t)).$
%				%			
%			\end{center}
%			%	
%			  \item 	This is equivalent to 
%			%	
%			\begin{align*}
%				\reg(a_0) - \reg(\tilde a)+ \sum\nolimits_{t=1}^T (\l(a_t,z_t) &- \l(\tilde a,z_t)) \\ &\leq  \reg(a_0) - \reg(a_1) + \sum\nolimits_{t=1}^T (\l(a_t,z_t) - \l(a_{t+1},z_t)).
%				%			
%			\end{align*}		 
%			%			 
%			  \item Rearranging yields
%			%			 
%			\begin{align*}
%				R_T^{FTRL}(\tilde a) = \sum\nolimits_{t=1}^T (\l(a_t,z_t) &- \l(\tilde a,z_t)) \\ &\quad \leq \reg(\tilde a) - \reg(a_1) + \sum\nolimits_{t=1}^T (\l(a_t,z_t) - \l(a_{t+1},z_t)).
%				%		
%			\end{align*}
%			%	
%			\qed
%			%	
%			%			
%		\end{itemize}
%	\end{itemize}
%\end{frame}

\begin{frame} 
	\frametitle{FTRL for online linear optimization}
	\small
	\begin{itemize}
		%		
		\item In the following, we analyze the FTRL algorithm for the linear loss $\l(a,z)=a^\top z$ for online linear optimization (OLO) problems.
		 \item For this purpose,  the squared L2-norm regularization will be used:
		%
		\begin{equation*}
			%		 \label{eq_l2_reg}
			%
			\reg(a) = \frac{1}{2 \eta}  \norm{a}^2 = \frac{a^\top a}{2 \eta}   ,
			%
		\end{equation*} 
		%
		where $\eta$ is some positive scalar, the \emph{regularization magnitude.}
		%
		  \item {\visible<2->{  It is straightforward to compute that if $\Aspace = \R^d,$ then
		%
		$$ a_t^{\FTRL} = - \eta  \sum\nolimits_{s=1}^{t-1} z_s.$$}}
		%
		%
		\item {\visible<3->{Hence, in this case we have for the FTRL algorithm the following update rule 
		%
		\begin{equation*}
			%		\label{eq:forel_update}
			%	
			a_{t+1}^{\FTRL} = a_t^{\FTRL} - \eta \, z_t, \qquad t=1,\ldots,T-1. 
			%	
		\end{equation*} }}
	
		%	
		{\visible<4>{    \emph{Interpretation:}  $-z_t$ is the \emph{direction} in which the update of $a_t^{\FTRL}$ to $a_{t+1}^{\FTRL}$ is conducted with \emph{step size} $\eta$ in order to reduce the loss.}}
		%		
	\end{itemize}
\end{frame}

\begin{frame} 
	\frametitle{FTRL for OLO: Theoretical guarantees}
	\small
	\begin{itemize}
		
		\item \textbf{Proposition:}
		%	
		Using the FTRL algorithm with the squared L2-norm regularization on any online linear optimization (OLO) problem with $\Aspace \subset \mathbb{R}^d$ leads to a regret of FTRL with respect to any action $\tilde a \in \Aspace$  of
		%	 
		\begin{equation*}
			%	
			R_T^{FTRL}(\tilde a)  \leq \frac{1}{2\eta}  \norm{\tilde a}^2 +   \eta  \sum\limits_{t=1}^T \norm{z_t}^2.
			%		
		\end{equation*}
		%		
		%		where $C>0$ is some constant depending on $\Aspace.$
		%	
		{\visible<2->{  \item We will show the result only for the case $\Aspace=\R^d.$   
		%	
		\item For the more general case, where $\Aspace$ is a strict subset of $\R^d,$ we need a slight modification of the update formula above:
		 %	
		 $$ a_t^{\FTRL} = \Pi_\Aspace\big( - \eta  \sum\nolimits_{i=1}^{t-1} z_i\big)  = \argmin_{a \in \mathcal{A}} \norm{ a - \eta   \sum\nolimits_{i=1}^{t-1} z_i }^2. $$
		 %	
		 In words, the action of the FTRL algorithm has to be projected onto the set $\Aspace.$
		 %	
		 %	Completing the square
		 Here, $\Pi_\Aspace: \R^d \to \Aspace$ is the projection onto $\Aspace.$
		 
		 {\tiny (The proof is essentially the same, except that the Cauchy-Schwarz inequality is used in between.)} }}
		%
	\end{itemize}
\end{frame}

\begin{frame} 
	\frametitle{FTRL for OLO: Theoretical guarantees}
	\small
	\begin{itemize}
		\footnotesize

		\item \textbf{Proof:}
		
		\fbox{\begin{minipage}{0.95\textwidth}
				\begin{equation*}
					%	
					\begin{split}
						% 
						& \mbox{\textbf{Reminder (1):} \quad }	R_T^{\FTRL}(\tilde a) \leq \reg(\tilde a) - \reg(a_1^{\FTRL}) +\sum\limits_{t=1}^T \left(\l(a_t^{\FTRL},z_t) - \l(a_{t+1}^{\FTRL},z_t)\right). \\
						%							
						& \mbox{\textbf{Reminder (2):} \quad }	a_{t+1}^{\FTRL} = a_t^{\FTRL} - \eta \, z_t, \qquad t=1,\ldots,T-1.
						%
					\end{split}
					%	
				\end{equation*}
			\end{minipage}
		}
		\footnotesize
			
			%	
			 \item 	For sake of brevity, we write $a_1, a_2, \ldots$ for $a_1^{\FTRL}, a_2^{\FTRL}, \ldots$ 
			%		
			\pause
			 \item With this,
			%
			\begin{align*}
				%				
				R_T^{FTRL}(\tilde a) &\leq \reg(\tilde a) - \reg(a_1) + \sum\nolimits_{t=1}^T (\l(a_t,z_t) - \l(a_{t+1},z_t))  \tag{Reminder (1)}\\
				%
				 &\leq \frac{1}{2\eta}  \norm{\tilde a}^2 + \sum\nolimits_{t=1}^T ( a_t^\top z_t  -  a_{t+1}^\top z_t  ) \tag{$\reg(a_1)\geq 0$ and definition of $\reg$} \\
				%
				 &= \frac{1}{2\eta}  \norm{\tilde a}^2 + \sum\nolimits_{t=1}^T  ( a_t^\top-  a_{t+1}^\top   )z_t \tag{Distributivity} \\
				%
				 &= \frac{1}{2\eta}  \norm{\tilde a}^2 + \eta \sum\nolimits_{t=1}^T   \norm{z_t}^2.  \tag{Reminder (2)}
				% 
				%
			\end{align*}
			%
			\qed
%		
	\end{itemize}
\end{frame}

\begin{frame} 
	\frametitle{FTRL for OLO: Theoretical guarantees}
	\small
	\begin{itemize}	 
		
		\footnotesize
		
		\item Interpretation of the terms in the proposition, i.e., of 
%		
		$$R_T^{FTRL}(\tilde a) \leq  \frac{1}{2\eta}  {\color{blue} \norm{\tilde a}^2 }+  \eta  {\color{orange} \sum\limits_{t=1}^T \norm{z_t}^2}:$$
		%	
		
		\begin{itemize}\footnotesize
			%	
			 \item  {\visible<2->{ $ {\color{blue} \norm{\tilde a}^2 }$ represents a  {\color{blue} \emph{bias term:}} The regret upper bound of FTRL is always biased by the term $ \norm{\tilde a}^2.$
			%	
			The impact of the bias term can be reduced by a higher regularization magnitude, i.e.,  a higher choice of $\eta.$ }}
			%	
			 \item  {\visible<3->{ ${\color{orange} \sum\limits_{t=1}^T \norm{z_t}^2}$ represents a  {\color{orange} \emph{''variance'' term}}: The more the environment data $z_t$ varies, the larger this term. Hence, for a high variance a smaller regularization magnitude is needed, i.e., a smaller choice of $\eta.$  }}
			
			
			%	
		\end{itemize}	
		
		%
		 \item  {\visible<4->{ 
		Thus, we have a trade-off for the optimal choice of $\eta:$ Making $\eta$ large, leads to a smaller {\color{blue} bias} but at the expense  of a higher {\color{orange} variance} and making $\eta$ small leads to a smaller {\color{orange} variance} at the expense  of a higher {\color{blue} bias}.}}
		
		 \item [$\Rightarrow$]  {\visible<5>{ With the right choice of $\eta$, we can prevent the instability of FTRL for an online linear optimization (OLO) problem. }}
		
		
		%	
	\end{itemize}
\end{frame}

\begin{frame} 
	\frametitle{FTRL for OLO: Theoretical guarantees}
	\small
	\begin{itemize}	 
		%
		\small	
		%
		
		\begin{minipage}{.5\textwidth}
			\item 
			Under certain assumptions we can balance the trade-off induced by the bias and the variance by choosing $\eta$ appropriately.
			
		\end{minipage}
		\begin{minipage}{.4\textwidth}
			
			\begin{figure}
				\centering
				\includegraphics[width=0.9\linewidth]{figure/bias-variance-tradeoff} 
			\end{figure}
			
		\end{minipage}
		%	
		%		
		 \item  {\visible<2->{ \textbf{Corollary:}
		%	
		Suppose we use the FTRL algorithm with the squared L2-norm regularization on an online linear optimization problem with $\Aspace \subset \mathbb{R}^d$ such that 
%		
		\begin{itemize}\small
%			
			\item $\sup_{\tilde a \in \Aspace}\norm{\tilde a} \leq B$ for some finite constant $B>0,$ 
%			
			\item $\sup_{z \in \Zspace}\norm{z} \leq V$ for some finite constant $V>0.$
%			
		\end{itemize}}}
		%	
		{\visible<3->{ Then, by choosing the step size $\eta$ for FTRL as $\eta = \frac{B}{V\sqrt{2\, T}}$ it holds that
		%	
			$$	R_T^{FTRL} \leq   BV\sqrt{2\, T}.		$$
			%	
		 }}
		%		
		\item  {\visible<4>{ Note that the (optimal) parameter $\eta$ depends on the time horizon $T,$ which is oftentimes not known in advance.
		%	
		However, there are some tricks (i.e., the \emph{doubling trick}), which can help in such cases. }}
		%	provides an algorithm which has the same order of the regret, but does not need the knowledge of the time horizon in advance.
		%		
	\end{itemize}
\end{frame}

\begin{frame} 
	\frametitle{FTRL for OLO: Theoretical guarantees}
	\small
	\begin{itemize}	 
		\small
		\item \textbf{Proof:}
		
		\begin{itemize} \small
			%	
			\item By the latter {\color{green} proposition} and the {\color{olive} assumptions }
			%	
			\begin{align*}
				%	
				R_T^{FTRL}(\tilde a)  
				%				 
				 \quad &{\color{green}\leq} \quad  \frac{1}{2\eta}  \norm{\tilde a}^2 \ &&+  \eta  \sum\limits_{t=1}^T \norm{z_t}^2 \\
				%				 
				%				&\leq \frac{1}{2\eta}  \underbrace{\norm{\tilde a}^2}_{\mbox{$\leq B^2$ by assumption}} \ +  \eta  \sum\limits_{t=1}^T \underbrace{\norm{z_t}^2 }_{\mbox{$\leq V^2$ by assumption}} \\
				%		
				 &{\color{olive} \leq} \quad \frac{B^2}{2\eta} \qquad \qquad\qquad\qquad\quad  &&+ \eta \, T \, V^2.
				%		
			\end{align*}
			%	
			 {\visible<2->{	\item The right-hand side of the latter display is independent of $\tilde a,$ so that 
			%	
			\begin{align*}
				%	
				R_T^{FTRL}	\leq   \frac{B^2}{2\eta}  +  \eta \, T \, V^2.
				%		
			\end{align*} }}
			%
			  {\visible<3->{ \item Now, the right-hand side of the latter display is a function of the form $f(\eta) = a/\eta + b \eta$ for some suitable $a,b>0.$ }}
			%	
			 {\visible<4->{ \item   Minimizing $f$ with respect to $\eta$ results in the minimizer $\eta^* = \frac{B}{V\sqrt{2\,T}}.$ }}
			%	
			  {\visible<5->{ \item Plugging this minimizer into the latter display leads to the asserted inequality. \qed }}	
			%	
		\end{itemize}
		%
	\end{itemize}
	%
\end{frame}

\begin{frame}
	\frametitle{Desired results}
	%	
	\small
	\begin{itemize}\small
		%		
		\item With the FTRL algorithm we can cope with 
%		
		\begin{itemize}\small
			%			
			 \item online quadratic optimization (OQO) problems by using no regularity ($\reg \equiv 0$). In this case, we have satisfactory regret guarantees and also a quick update rule for $a_{t+1}^{\FTRL}$ (It is just the empirical average over all data points seen till $t$),
			%			
			 {\visible<2->{  \item online linear optimization (OLO) problems by using a suitable regularization function.
			%			
			In this case, we have quick update formulas and satisfactory regret guarantees as well. }}
			%
		\end{itemize}
		%		
		   \item [$\Rightarrow$]  {\visible<3->{ But what about other online learning problems or rather other loss functions? }}
		%
		 {\visible<3>{   \item  What we wish to have is an approach such that we can achieve for a large class of loss functions $\l$ the advantages of FTRL for OLO and OCO problems:
		
		\begin{itemize}\small
			 \item  [(a)] reasonable regret upper bounds;
			  \item  [(b)] a quick update formula.
		\end{itemize} }}
		%		
%		 \item For this purpose, we will dig into the theory of \emph{convex functions} in order to transfer the advantages of FTRL to a larger class of online learning problems.
		%
	\end{itemize}
	%	
\end{frame}




%
\endlecture
\end{document}
