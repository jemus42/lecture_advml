\documentclass[11pt,compress,t,notes=noshow, xcolor=table]{beamer}
\usepackage[]{graphicx}
\usepackage[]{color}
% maxwidth is the original width if it is less than linewidth
% otherwise use linewidth (to make sure the graphics do not exceed the margin)
\makeatletter
\def\maxwidth{ %
  \ifdim\Gin@nat@width>\linewidth
    \linewidth
  \else
    \Gin@nat@width
  \fi
}
\makeatother

% ---------------------------------%
% latex-math dependencies, do not remove:
% - \usepackage{mathtools}
% - \usepackage{bm}
% - \usepackage{siunitx}
% - \usepackage{dsfont}
% - \usepackage{xspace}
% ---------------------------------%

%--------------------------------------------------------%
%       Language, encoding, typography
%--------------------------------------------------------%

\usepackage[english]{babel}
\usepackage[utf8]{inputenc} % Enables inputting UTF-8 symbols
% Standard AMS suite
\usepackage{amsmath,amsfonts,amssymb}

% Font four double-stroke / blackboard letters for sets of numbers (N, R, ...)
% Distribution name is "doublestroke"
% According to https://mirror.physik.tu-berlin.de/pub/CTAN/fonts/doublestroke/dsdoc.pdf
% the "bbm" package does a similar thing and may be superfluous.
% Required for latex-math
\usepackage{dsfont}

% bbm – "Blackboard-style" cm fonts (https://www.ctan.org/pkg/bbm)
% Used to be in common.tex, loaded directly after this file
% Maybe superfluous given dsfont is loaded
% TODO: Check if really unused?
% \usepackage{bbm}

% bm – Access bold symbols in maths mode - https://ctan.org/pkg/bm
% Required for latex-math
% https://tex.stackexchange.com/questions/3238/bm-package-versus-boldsymbol
\usepackage{bm}

% pifont – Access to PostScript standard Symbol and Dingbats fonts
% Used for \newcommand{\xmark}{\ding{55}, which is never used
% aside from lecture_advml/attic/xx-automl/slides.Rnw
% \usepackage{pifont}

% Quotes (inline and display), provdes \enquote
% https://ctan.org/pkg/csquotes
\usepackage{csquotes}

% Adds arg to enumerate env, technically superseded by enumitem according
% to https://ctan.org/pkg/enumerate
% Replace with https://ctan.org/pkg/enumitem ?
\usepackage{enumerate}

% Line spacing - provides \singlespacing \doublespacing \onehalfspacing
% https://ctan.org/pkg/setspace
% TODO: Check if really unused?
%\usepackage{setspace}

% mathtools – Mathematical tools to use with amsmath
% https://ctan.org/pkg/mathtools?lang=en
% latex-math dependency according to latex-math repo
\usepackage{mathtools}

%--------------------------------------------------------%
%       Displaying code and algorithms
%--------------------------------------------------------%
\usepackage{verbatim}
\usepackage{algorithm}
\usepackage{algpseudocode}

%--------------------------------------------------------%
%       Tables
%--------------------------------------------------------%

% multi-row table cells: https://www.namsu.de/Extra/pakete/Multirow.html
\usepackage{multirow}

% long/multi-page tables: https://texdoc.org/serve/longtable.pdf/0
% TODO: Check if really unused?

\usepackage{longtable}

% pretty table env: https://ctan.org/pkg/booktabs?lang=en
% TODO: Check if really unused?
\usepackage{booktabs}

%--------------------------------------------------------%
%       Figures: Creating, placing, verbing
%--------------------------------------------------------%

% wrapfig - Wrapping text around figures https://de.overleaf.com/learn/latex/Wrapping_text_around_figures
\usepackage{wrapfig}

% Sub figures in figures and tables
% https://ctan.org/pkg/subfig -- supersedes subfigure package
% TODO: Check if really unused?
\usepackage{subfig}

% Actually it's pronounced PGF https://en.wikibooks.org/wiki/LaTeX/PGF/TikZ
\usepackage{tikz}

\usetikzlibrary{shapes,arrows,automata,positioning,calc,chains,trees, shadows}
\tikzset{
  %Define standard arrow tip
  >=stealth',
  %Define style for boxes
  punkt/.style={
    rectangle,
    rounded corners,
    draw=black, very thick,
    text width=6.5em,
    minimum height=2em,
    text centered},
  % Define arrow style
  pil/.style={
    ->,
    thick,
    shorten <=2pt,
    shorten >=2pt,}
}


% Unsorted
% textpos – Place boxes at arbitrary positions on the LATEX page
% https://ctan.org/pkg/textpos?lang=en
% Provides \begin{textblock}
 % TODO: Check if really unused?
\usepackage[absolute,overlay]{textpos}

% psfrag – Replace strings in encapsulated PostScript figures
% https://www.overleaf.com/latex/examples/psfrag-example/tggxhgzwrzhn
% https://ftp.mpi-inf.mpg.de/pub/tex/mirror/ftp.dante.de/pub/tex/macros/latex/contrib/psfrag/pfgguide.pdf
% Can't tell if this is needed
% TODO: Check if really unused?
\usepackage{psfrag}

% Maybe not great to use this https://tex.stackexchange.com/a/197/19093
% Use align instead -- TODO: Global search & replace to check
\usepackage{eqnarray}

\usepackage{colortbl}

% arydshln – Draw dash-lines in array/tabular
% https://www.ctan.org/pkg/arydshln
% !! "arydshln has to be loaded after array, longtable, colortab and/or colortbl"
% Provides \hdashline and \cdashline
% TODO: Check if really unused?
% \usepackage{arydshln}

% tabularx – Tabulars with adjustable-width columns
% https://ctan.org/pkg/tabularx
% Provides \begin{tabularx}
% TODO: Check if really unused?
% \usepackage{tabularx}

% placeins – Control float placement
% https://ctan.org/pkg/placeins
% Defines a \FloatBarrier command
% TODO: Check if really unused?
% \usepackage{placeins}


% framed – Framed or shaded regions that can break across pages
% https://ctan.org/pkg/framed
% Provides \begin{framed} which uses \colorbox{shadecolor} relying on \definecolor{shadecolor}.
% TODO: Check if really unused?
% \usepackage{framed}

% Used often in conjunction with \definecolor{shadecolor}{rgb}{0.969, 0.969, 0.969}
% Might be able to be removed or at least redefined to only have shadecolor (if needed)
\definecolor{fgcolor}{rgb}{0.345, 0.345, 0.345}
\definecolor{shadecolor}{rgb}{0.969, 0.969, 0.969}
\newenvironment{knitrout}{}{} % an empty environment to be redefined in TeX


% Defines macros and environments
\usepackage{../../style/lmu-lecture}

\let\code=\texttt % Used regularly
\let\proglang=\textsf % Unused?

% Not sure what/why this does
\setkeys{Gin}{width=0.9\textwidth}

\setbeamertemplate{frametitle}{\expandafter\uppercase\expandafter\insertframetitle}

% Can't find a reason why common.tex is not just part of this file?
% Rarely used fontstyle for R packages, used only in 
% - forests/slides-forests-benchmark.tex
% - exercises/single-exercises/methods_l_1.Rnw
% - slides/cart/attic/slides_extra_trees.Rnw
\newcommand{\pkg}[1]{{\fontseries{b}\selectfont #1}}

% Spacing helpers, used often (mostly in exercises for \dlz)
\newcommand{\lz}{\vspace{0.5cm}} % vertical space (used often in slides)
\newcommand{\dlz}{\vspace{1cm}}  % double vertical space (used often in exercises, never in slides)

%--------------------%
%  New environments  %
%--------------------%

 % Frame with breaks and verbatim // this is used very often
\newenvironment{vbframe}
{
\begin{frame}[containsverbatim,allowframebreaks]
}
{
\end{frame}
}

% Frame with verbatim without breaks (to avoid numbering one slided frames)
% This is not used anywhere but I can see it being useful
\newenvironment{vframe}
{
\begin{frame}[containsverbatim]
}
{
\end{frame}
}

% Itemize block
\newenvironment{blocki}[1]
{
\begin{block}{#1}\begin{itemize}
}
{
\end{itemize}\end{block}
}

% textcolor that works in mathmode
% https://tex.stackexchange.com/a/261480
% Used e.g. in forests/slides-forests-bagging.tex
% [...] \textcolor{blue}{\tfrac{1}{M}\sum^M_{m} [...]
% \makeatletter
% \renewcommand*{\@textcolor}[3]{%
%   \protect\leavevmode
%   \begingroup
%     \color#1{#2}#3%
%   \endgroup
% }
% \makeatother


%-------------------------------------------------------------------------------------------------------%
%  Unused stuff that needs to go but is kept here currently juuuust in case it was important after all  %
%-------------------------------------------------------------------------------------------------------%

% \newcommand{\hlnum}[1]{\textcolor[rgb]{0.686,0.059,0.569}{#1}}%
% \newcommand{\hlstr}[1]{\textcolor[rgb]{0.192,0.494,0.8}{#1}}%
% \newcommand{\hlcom}[1]{\textcolor[rgb]{0.678,0.584,0.686}{\textit{#1}}}%
% \newcommand{\hlopt}[1]{\textcolor[rgb]{0,0,0}{#1}}%
% \newcommand{\hlstd}[1]{\textcolor[rgb]{0.345,0.345,0.345}{#1}}%
% \newcommand{\hlkwa}[1]{\textcolor[rgb]{0.161,0.373,0.58}{\textbf{#1}}}%
% \newcommand{\hlkwb}[1]{\textcolor[rgb]{0.69,0.353,0.396}{#1}}%
% \newcommand{\hlkwc}[1]{\textcolor[rgb]{0.333,0.667,0.333}{#1}}%
% \newcommand{\hlkwd}[1]{\textcolor[rgb]{0.737,0.353,0.396}{\textbf{#1}}}%
% \let\hlipl\hlkwb

% \makeatletter
% \newenvironment{kframe}{%
%  \def\at@end@of@kframe{}%
%  \ifinner\ifhmode%
%   \def\at@end@of@kframe{\end{minipage}}%
%   \begin{minipage}{\columnwidth}%
%  \fi\fi%
%  \def\FrameCommand##1{\hskip\@totalleftmargin \hskip-\fboxsep
%  \colorbox{shadecolor}{##1}\hskip-\fboxsep
%      % There is no \\@totalrightmargin, so:
%      \hskip-\linewidth \hskip-\@totalleftmargin \hskip\columnwidth}%
%  \MakeFramed {\advance\hsize-\width
%    \@totalleftmargin\z@ \linewidth\hsize
%    \@setminipage}}%
%  {\par\unskip\endMakeFramed%
%  \at@end@of@kframe}
% \makeatother

% \definecolor{shadecolor}{rgb}{.97, .97, .97}
% \definecolor{messagecolor}{rgb}{0, 0, 0}
% \definecolor{warningcolor}{rgb}{1, 0, 1}
% \definecolor{errorcolor}{rgb}{1, 0, 0}
% \newenvironment{knitrout}{}{} % an empty environment to be redefined in TeX

% \usepackage{alltt}
% \newcommand{\SweaveOpts}[1]{}  % do not interfere with LaTeX
% \newcommand{\SweaveInput}[1]{} % because they are not real TeX commands
% \newcommand{\Sexpr}[1]{}       % will only be parsed by R
% \newcommand{\xmark}{\ding{55}}%

% dependencies: amsmath, amssymb, dsfont
% math spaces
\ifdefined\N
\renewcommand{\N}{\mathds{N}} % N, naturals
\else \newcommand{\N}{\mathds{N}} \fi
\newcommand{\Z}{\mathds{Z}} % Z, integers
\newcommand{\Q}{\mathds{Q}} % Q, rationals
\newcommand{\R}{\mathds{R}} % R, reals
\ifdefined\C
\renewcommand{\C}{\mathds{C}} % C, complex
\else \newcommand{\C}{\mathds{C}} \fi
\newcommand{\continuous}{\mathcal{C}} % C, space of continuous functions
\newcommand{\M}{\mathcal{M}} % machine numbers
\newcommand{\epsm}{\epsilon_m} % maximum error

% counting / finite sets
\newcommand{\setzo}{\{0, 1\}} % set 0, 1
\newcommand{\setmp}{\{-1, +1\}} % set -1, 1
\newcommand{\unitint}{[0, 1]} % unit interval

% basic math stuff
\newcommand{\xt}{\tilde x} % x tilde
\DeclareMathOperator*{\argmax}{arg\,max} % argmax
\DeclareMathOperator*{\argmin}{arg\,min} % argmin
\newcommand{\argminlim}{\mathop{\mathrm{arg\,min}}\limits} % argmax with limits
\newcommand{\argmaxlim}{\mathop{\mathrm{arg\,max}}\limits} % argmin with limits
\newcommand{\sign}{\operatorname{sign}} % sign, signum
\newcommand{\I}{\mathbb{I}} % I, indicator
\newcommand{\order}{\mathcal{O}} % O, order
\newcommand{\bigO}{\mathcal{O}} % Big-O Landau
\newcommand{\littleo}{{o}} % Little-o Landau
\newcommand{\pd}[2]{\frac{\partial{#1}}{\partial #2}} % partial derivative
\newcommand{\floorlr}[1]{\left\lfloor #1 \right\rfloor} % floor
\newcommand{\ceillr}[1]{\left\lceil #1 \right\rceil} % ceiling
\newcommand{\indep}{\perp \!\!\! \perp} % independence symbol

% sums and products
\newcommand{\sumin}{\sum\limits_{i=1}^n} % summation from i=1 to n
\newcommand{\sumim}{\sum\limits_{i=1}^m} % summation from i=1 to m
\newcommand{\sumjn}{\sum\limits_{j=1}^n} % summation from j=1 to p
\newcommand{\sumjp}{\sum\limits_{j=1}^p} % summation from j=1 to p
\newcommand{\sumik}{\sum\limits_{i=1}^k} % summation from i=1 to k
\newcommand{\sumkg}{\sum\limits_{k=1}^g} % summation from k=1 to g
\newcommand{\sumjg}{\sum\limits_{j=1}^g} % summation from j=1 to g
\newcommand{\meanin}{\frac{1}{n} \sum\limits_{i=1}^n} % mean from i=1 to n
\newcommand{\meanim}{\frac{1}{m} \sum\limits_{i=1}^m} % mean from i=1 to n
\newcommand{\meankg}{\frac{1}{g} \sum\limits_{k=1}^g} % mean from k=1 to g
\newcommand{\prodin}{\prod\limits_{i=1}^n} % product from i=1 to n
\newcommand{\prodkg}{\prod\limits_{k=1}^g} % product from k=1 to g
\newcommand{\prodjp}{\prod\limits_{j=1}^p} % product from j=1 to p

% linear algebra
\newcommand{\one}{\bm{1}} % 1, unitvector
\newcommand{\zero}{\mathbf{0}} % 0-vector
\newcommand{\id}{\bm{I}} % I, identity
\newcommand{\diag}{\operatorname{diag}} % diag, diagonal
\newcommand{\trace}{\operatorname{tr}} % tr, trace
\newcommand{\spn}{\operatorname{span}} % span
\newcommand{\scp}[2]{\left\langle #1, #2 \right\rangle} % <.,.>, scalarproduct
\newcommand{\mat}[1]{\begin{pmatrix} #1 \end{pmatrix}} % short pmatrix command
\newcommand{\Amat}{\mathbf{A}} % matrix A
\newcommand{\Deltab}{\mathbf{\Delta}} % error term for vectors

% basic probability + stats
\renewcommand{\P}{\mathds{P}} % P, probability
\newcommand{\E}{\mathds{E}} % E, expectation
\newcommand{\var}{\mathsf{Var}} % Var, variance
\newcommand{\cov}{\mathsf{Cov}} % Cov, covariance
\newcommand{\corr}{\mathsf{Corr}} % Corr, correlation
\newcommand{\normal}{\mathcal{N}} % N of the normal distribution
\newcommand{\iid}{\overset{i.i.d}{\sim}} % dist with i.i.d superscript
\newcommand{\distas}[1]{\overset{#1}{\sim}} % ... is distributed as ...

% machine learning
\newcommand{\Xspace}{\mathcal{X}} % X, input space
\newcommand{\Yspace}{\mathcal{Y}} % Y, output space
\newcommand{\Zspace}{\mathcal{Z}} % Space of sampled datapoints ! Also defined identically in ml-online.tex !
\newcommand{\nset}{\{1, \ldots, n\}} % set from 1 to n
\newcommand{\pset}{\{1, \ldots, p\}} % set from 1 to p
\newcommand{\gset}{\{1, \ldots, g\}} % set from 1 to g
\newcommand{\Pxy}{\mathbb{P}_{xy}} % P_xy
\newcommand{\Exy}{\mathbb{E}_{xy}} % E_xy: Expectation over random variables xy
\newcommand{\xv}{\mathbf{x}} % vector x (bold)
\newcommand{\xtil}{\tilde{\mathbf{x}}} % vector x-tilde (bold)
\newcommand{\yv}{\mathbf{y}} % vector y (bold)
\newcommand{\xy}{(\xv, y)} % observation (x, y)
\newcommand{\xvec}{\left(x_1, \ldots, x_p\right)^\top} % (x1, ..., xp)
\newcommand{\Xmat}{\mathbf{X}} % Design matrix
\newcommand{\allDatasets}{\mathds{D}} % The set of all datasets
\newcommand{\allDatasetsn}{\mathds{D}_n}  % The set of all datasets of size n
\newcommand{\D}{\mathcal{D}} % D, data
\newcommand{\Dn}{\D_n} % D_n, data of size n
\newcommand{\Dtrain}{\mathcal{D}_{\text{train}}} % D_train, training set
\newcommand{\Dtest}{\mathcal{D}_{\text{test}}} % D_test, test set
\newcommand{\xyi}[1][i]{\left(\xv^{(#1)}, y^{(#1)}\right)} % (x^i, y^i), i-th observation
\newcommand{\Dset}{\left( \xyi[1], \ldots, \xyi[n]\right)} % {(x1,y1)), ..., (xn,yn)}, data
\newcommand{\defAllDatasetsn}{(\Xspace \times \Yspace)^n} % Def. of the set of all datasets of size n
\newcommand{\defAllDatasets}{\bigcup_{n \in \N}(\Xspace \times \Yspace)^n} % Def. of the set of all datasets
\newcommand{\xdat}{\left\{ \xv^{(1)}, \ldots, \xv^{(n)}\right\}} % {x1, ..., xn}, input data
\newcommand{\ydat}{\left\{ \yv^{(1)}, \ldots, \yv^{(n)}\right\}} % {y1, ..., yn}, input data
\newcommand{\yvec}{\left(y^{(1)}, \hdots, y^{(n)}\right)^\top} % (y1, ..., yn), vector of outcomes
\renewcommand{\xi}[1][i]{\xv^{(#1)}} % x^i, i-th observed value of x
\newcommand{\yi}[1][i]{y^{(#1)}} % y^i, i-th observed value of y
\newcommand{\xivec}{\left(x^{(i)}_1, \ldots, x^{(i)}_p\right)^\top} % (x1^i, ..., xp^i), i-th observation vector
\newcommand{\xj}{\xv_j} % x_j, j-th feature
\newcommand{\xjvec}{\left(x^{(1)}_j, \ldots, x^{(n)}_j\right)^\top} % (x^1_j, ..., x^n_j), j-th feature vector
\newcommand{\phiv}{\mathbf{\phi}} % Basis transformation function phi
\newcommand{\phixi}{\mathbf{\phi}^{(i)}} % Basis transformation of xi: phi^i := phi(xi)

%%%%%% ml - models general
\newcommand{\lamv}{\bm{\lambda}} % lambda vector, hyperconfiguration vector
\newcommand{\Lam}{\bm{\Lambda}}	 % Lambda, space of all hpos
% Inducer / Inducing algorithm
\newcommand{\preimageInducer}{\left(\defAllDatasets\right)\times\Lam} % Set of all datasets times the hyperparameter space
\newcommand{\preimageInducerShort}{\allDatasets\times\Lam} % Set of all datasets times the hyperparameter space
% Inducer / Inducing algorithm
\newcommand{\ind}{\mathcal{I}} % Inducer, inducing algorithm, learning algorithm

% continuous prediction function f
\newcommand{\ftrue}{f_{\text{true}}}  % True underlying function (if a statistical model is assumed)
\newcommand{\ftruex}{\ftrue(\xv)} % True underlying function (if a statistical model is assumed)
\newcommand{\fx}{f(\xv)} % f(x), continuous prediction function
\newcommand{\fdomains}{f: \Xspace \rightarrow \R^g} % f with domain and co-domain
\newcommand{\Hspace}{\mathcal{H}} % hypothesis space where f is from
\newcommand{\fbayes}{f^{\ast}} % Bayes-optimal model
\newcommand{\fxbayes}{f^{\ast}(\xv)} % Bayes-optimal model
\newcommand{\fkx}[1][k]{f_{#1}(\xv)} % f_j(x), discriminant component function
\newcommand{\fh}{\hat{f}} % f hat, estimated prediction function
\newcommand{\fxh}{\fh(\xv)} % fhat(x)
\newcommand{\fxt}{f(\xv ~|~ \thetab)} % f(x | theta)
\newcommand{\fxi}{f\left(\xv^{(i)}\right)} % f(x^(i))
\newcommand{\fxih}{\hat{f}\left(\xv^{(i)}\right)} % f(x^(i))
\newcommand{\fxit}{f\left(\xv^{(i)} ~|~ \thetab\right)} % f(x^(i) | theta)
\newcommand{\fhD}{\fh_{\D}} % fhat_D, estimate of f based on D
\newcommand{\fhDtrain}{\fh_{\Dtrain}} % fhat_Dtrain, estimate of f based on D
\newcommand{\fhDnlam}{\fh_{\Dn, \lamv}} %model learned on Dn with hp lambda
\newcommand{\fhDlam}{\fh_{\D, \lamv}} %model learned on D with hp lambda
\newcommand{\fhDnlams}{\fh_{\Dn, \lamv^\ast}} %model learned on Dn with optimal hp lambda
\newcommand{\fhDlams}{\fh_{\D, \lamv^\ast}} %model learned on D with optimal hp lambda

% discrete prediction function h
\newcommand{\hx}{h(\xv)} % h(x), discrete prediction function
\newcommand{\hh}{\hat{h}} % h hat
\newcommand{\hxh}{\hat{h}(\xv)} % hhat(x)
\newcommand{\hxt}{h(\xv | \thetab)} % h(x | theta)
\newcommand{\hxi}{h\left(\xi\right)} % h(x^(i))
\newcommand{\hxit}{h\left(\xi ~|~ \thetab\right)} % h(x^(i) | theta)
\newcommand{\hbayes}{h^{\ast}} % Bayes-optimal classification model
\newcommand{\hxbayes}{h^{\ast}(\xv)} % Bayes-optimal classification model

% yhat
\newcommand{\yh}{\hat{y}} % yhat for prediction of target
\newcommand{\yih}{\hat{y}^{(i)}} % yhat^(i) for prediction of ith targiet
\newcommand{\resi}{\yi- \yih}

% theta
\newcommand{\thetah}{\hat{\theta}} % theta hat
\newcommand{\thetab}{\bm{\theta}} % theta vector
\newcommand{\thetabh}{\bm{\hat\theta}} % theta vector hat
\newcommand{\thetat}[1][t]{\thetab^{[#1]}} % theta^[t] in optimization
\newcommand{\thetatn}[1][t]{\thetab^{[#1 +1]}} % theta^[t+1] in optimization
\newcommand{\thetahDnlam}{\thetabh_{\Dn, \lamv}} %theta learned on Dn with hp lambda
\newcommand{\thetahDlam}{\thetabh_{\D, \lamv}} %theta learned on D with hp lambda
\newcommand{\mint}{\min_{\thetab \in \Theta}} % min problem theta
\newcommand{\argmint}{\argmin_{\thetab \in \Theta}} % argmin theta

% densities + probabilities
% pdf of x
\newcommand{\pdf}{p} % p
\newcommand{\pdfx}{p(\xv)} % p(x)
\newcommand{\pixt}{\pi(\xv~|~ \thetab)} % pi(x|theta), pdf of x given theta
\newcommand{\pixit}[1][i]{\pi\left(\xi[#1] ~|~ \thetab\right)} % pi(x^i|theta), pdf of x given theta
\newcommand{\pixii}[1][i]{\pi\left(\xi[#1]\right)} % pi(x^i), pdf of i-th x

% pdf of (x, y)
\newcommand{\pdfxy}{p(\xv,y)} % p(x, y)
\newcommand{\pdfxyt}{p(\xv, y ~|~ \thetab)} % p(x, y | theta)
\newcommand{\pdfxyit}{p\left(\xi, \yi ~|~ \thetab\right)} % p(x^(i), y^(i) | theta)

% pdf of x given y
\newcommand{\pdfxyk}[1][k]{p(\xv | y= #1)} % p(x | y = k)
\newcommand{\lpdfxyk}[1][k]{\log p(\xv | y= #1)} % log p(x | y = k)
\newcommand{\pdfxiyk}[1][k]{p\left(\xi | y= #1 \right)} % p(x^i | y = k)

% prior probabilities
\newcommand{\pik}[1][k]{\pi_{#1}} % pi_k, prior
\newcommand{\lpik}[1][k]{\log \pi_{#1}} % log pi_k, log of the prior
\newcommand{\pit}{\pi(\thetab)} % Prior probability of parameter theta

% posterior probabilities
\newcommand{\post}{\P(y = 1 ~|~ \xv)} % P(y = 1 | x), post. prob for y=1
\newcommand{\postk}[1][k]{\P(y = #1 ~|~ \xv)} % P(y = k | y), post. prob for y=k
\newcommand{\pidomains}{\pi: \Xspace \rightarrow \unitint} % pi with domain and co-domain
\newcommand{\pibayes}{\pi^{\ast}} % Bayes-optimal classification model
\newcommand{\pixbayes}{\pi^{\ast}(\xv)} % Bayes-optimal classification model
\newcommand{\pix}{\pi(\xv)} % pi(x), P(y = 1 | x)
\newcommand{\piv}{\bm{\pi}} % pi, bold, as vector
\newcommand{\pikx}[1][k]{\pi_{#1}(\xv)} % pi_k(x), P(y = k | x)
\newcommand{\pikxt}[1][k]{\pi_{#1}(\xv ~|~ \thetab)} % pi_k(x | theta), P(y = k | x, theta)
\newcommand{\pixh}{\hat \pi(\xv)} % pi(x) hat, P(y = 1 | x) hat
\newcommand{\pikxh}[1][k]{\hat \pi_{#1}(\xv)} % pi_k(x) hat, P(y = k | x) hat
\newcommand{\pixih}{\hat \pi(\xi)} % pi(x^(i)) with hat
\newcommand{\pikxih}[1][k]{\hat \pi_{#1}(\xi)} % pi_k(x^(i)) with hat
\newcommand{\pdfygxt}{p(y ~|~\xv, \thetab)} % p(y | x, theta)
\newcommand{\pdfyigxit}{p\left(\yi ~|~\xi, \thetab\right)} % p(y^i |x^i, theta)
\newcommand{\lpdfygxt}{\log \pdfygxt } % log p(y | x, theta)
\newcommand{\lpdfyigxit}{\log \pdfyigxit} % log p(y^i |x^i, theta)

% probababilistic
\newcommand{\bayesrulek}[1][k]{\frac{\P(\xv | y= #1) \P(y= #1)}{\P(\xv)}} % Bayes rule
\newcommand{\muk}{\bm{\mu_k}} % mean vector of class-k Gaussian (discr analysis)

% residual and margin
\newcommand{\eps}{\epsilon} % residual, stochastic
\newcommand{\epsi}{\epsilon^{(i)}} % epsilon^i, residual, stochastic
\newcommand{\epsh}{\hat{\epsilon}} % residual, estimated
\newcommand{\yf}{y \fx} % y f(x), margin
\newcommand{\yfi}{\yi \fxi} % y^i f(x^i), margin
\newcommand{\Sigmah}{\hat \Sigma} % estimated covariance matrix
\newcommand{\Sigmahj}{\hat \Sigma_j} % estimated covariance matrix for the j-th class

% ml - loss, risk, likelihood
\newcommand{\Lyf}{L\left(y, f\right)} % L(y, f), loss function
\newcommand{\Lypi}{L\left(y, \pi\right)} % L(y, pi), loss function
\newcommand{\Lxy}{L\left(y, \fx\right)} % L(y, f(x)), loss function
\newcommand{\Lxyi}{L\left(\yi, \fxi\right)} % loss of observation
\newcommand{\Lxyt}{L\left(y, \fxt\right)} % loss with f parameterized
\newcommand{\Lxyit}{L\left(\yi, \fxit\right)} % loss of observation with f parameterized
\newcommand{\Lxym}{L\left(\yi, f\left(\bm{\tilde{x}}^{(i)} ~|~ \thetab\right)\right)} % loss of observation with f parameterized
\newcommand{\Lpixy}{L\left(y, \pix\right)} % loss in classification
\newcommand{\Lpiv}{L\left(y, \piv\right)} % loss in classification
\newcommand{\Lpixyi}{L\left(\yi, \pixii\right)} % loss of observation in classification
\newcommand{\Lpixyt}{L\left(y, \pixt\right)} % loss with pi parameterized
\newcommand{\Lpixyit}{L\left(\yi, \pixit\right)} % loss of observation with pi parameterized
\newcommand{\Lhxy}{L\left(y, \hx\right)} % L(y, h(x)), loss function on discrete classes
\newcommand{\Lr}{L\left(r\right)} % L(r), loss defined on residual (reg) / margin (classif)
\newcommand{\lone}{|y - \fx|} % L1 loss
\newcommand{\ltwo}{\left(y - \fx\right)^2} % L2 loss
\newcommand{\lbernoullimp}{\ln(1 + \exp(-y \cdot \fx))} % Bernoulli loss for -1, +1 encoding
\newcommand{\lbernoullizo}{- y \cdot \fx + \log(1 + \exp(\fx))} % Bernoulli loss for 0, 1 encoding
\newcommand{\lcrossent}{- y \log \left(\pix\right) - (1 - y) \log \left(1 - \pix\right)} % cross-entropy loss
\newcommand{\lbrier}{\left(\pix - y \right)^2} % Brier score
\newcommand{\risk}{\mathcal{R}} % R, risk
\newcommand{\riskbayes}{\mathcal{R}^\ast}
\newcommand{\riskf}{\risk(f)} % R(f), risk
\newcommand{\riskdef}{\E_{y|\xv}\left(\Lxy \right)} % risk def (expected loss)
\newcommand{\riskt}{\mathcal{R}(\thetab)} % R(theta), risk
\newcommand{\riske}{\mathcal{R}_{\text{emp}}} % R_emp, empirical risk w/o factor 1 / n
\newcommand{\riskeb}{\bar{\mathcal{R}}_{\text{emp}}} % R_emp, empirical risk w/ factor 1 / n
\newcommand{\riskef}{\riske(f)} % R_emp(f)
\newcommand{\risket}{\mathcal{R}_{\text{emp}}(\thetab)} % R_emp(theta)
\newcommand{\riskr}{\mathcal{R}_{\text{reg}}} % R_reg, regularized risk
\newcommand{\riskrt}{\mathcal{R}_{\text{reg}}(\thetab)} % R_reg(theta)
\newcommand{\riskrf}{\riskr(f)} % R_reg(f)
\newcommand{\riskrth}{\hat{\mathcal{R}}_{\text{reg}}(\thetab)} % hat R_reg(theta)
\newcommand{\risketh}{\hat{\mathcal{R}}_{\text{emp}}(\thetab)} % hat R_emp(theta)
\newcommand{\LL}{\mathcal{L}} % L, likelihood
\newcommand{\LLt}{\mathcal{L}(\thetab)} % L(theta), likelihood
\newcommand{\LLtx}{\mathcal{L}(\thetab | \xv)} % L(theta|x), likelihood
\newcommand{\logl}{\ell} % l, log-likelihood
\newcommand{\loglt}{\logl(\thetab)} % l(theta), log-likelihood
\newcommand{\logltx}{\logl(\thetab | \xv)} % l(theta|x), log-likelihood
\newcommand{\errtrain}{\text{err}_{\text{train}}} % training error
\newcommand{\errtest}{\text{err}_{\text{test}}} % test error
\newcommand{\errexp}{\overline{\text{err}_{\text{test}}}} % avg training error

% lm
\newcommand{\thx}{\thetab^\top \xv} % linear model
\newcommand{\olsest}{(\Xmat^\top \Xmat)^{-1} \Xmat^\top \yv} % OLS estimator in LM


\newcommand{\sens}{\mathbf{A}} % vector x (bold)
\newcommand{\ba}{\mathbf{a}}
\newcommand{\batilde}{\tilde{\mathbf{a}}}
\newcommand{\Px}{\mathbb{P}_{x}} % P_x
\newcommand{\Pxj}{\mathbb{P}_{x_j}} % P_{x_j}
\newcommand{\indep}{\perp \!\!\! \perp} % independence symbol
% ml - ROC
\newcommand{\np}{n_{+}} % no. of positive instances
\newcommand{\nn}{n_{-}} % no. of negative instances
\newcommand{\rn}{\pi_{-}} % proportion negative instances
\newcommand{\rp}{\pi_{+}} % proportion negative instances
% true/false pos/neg:
\newcommand{\tp}{\# \text{TP}} % true pos
\newcommand{\fap}{\# \text{FP}} % false pos (fp taken for partial derivs)
\newcommand{\tn}{\# \text{TN}} % true neg
\newcommand{\fan}{\# \text{FN}} % false neg

\newcommand{\Tspace}{\mathcal{T}}
\newcommand{\tv}{\mathbf{t}}
\newcommand{\tim}{\mathbf{t}^{(i)}_m}
\newcommand{\yim}{y^{(i)}_m}

\usepackage{multicol}

\newcommand{\titlefigure}{figure/Slide2}
\newcommand{\learninggoals}{
  \item Understand the practical relevance of multi-target prediction problems
  \item Know the relevant special cases of multi-target prediction
  \item Understand the difference between inductive and transductive learning problems
}

\title{Advanced Machine Learning}
\date{}

\begin{document}

\lecturechapter{Multi-Target Prediction: Introduction}
\lecture{Advanced Machine Learning}



\sloppy

\begin{vbframe}{Multi-target prediction: Motivation}
\scriptsize{
    \begin{itemize}    
        \item Conventional supervised learning: Label/Outcome space $\Yspace$ is one-dimensional.	

            \item Multi-target prediction (MTP) problems: multiple target variables of possibly different type (binary, nominal, ordinal, real-valued).
            
            \item Why not just \emph{reducing} an MTP problem to learning one model per target, independently of the other targets? 
        \hfill

		\item $\leadsto$ The targets can be more or less \emph{similar} to each other and exhibit \emph{statistical dependencies}.

        \item Consider the multi-label emotions dataset, where emotions of music pieces are automatically detected. Multiple emotions may be attributed to a single piece. We calculate the mutual information of the labels, which are clearly stochastically dependent:
        \vspace{10pt}
        \begin{center}
            \begin{tabular}{lrrrr}
\toprule
  & \textbf{Calm} & \textbf{Quiet} & \textbf{Sad} & \textbf{Angry}\\
\midrule
\textbf{Calm} & 1.000 & 0.073 & 0.018 & \textcolor{red}{\textbf{0.290}}\\
\textbf{Quiet} & 0.073 & 1.000 & \textcolor{red}{\textbf{0.241}} & \textcolor{red}{\textbf{0.164}}\\
\textbf{Sad} & 0.018 & \textcolor{red}{\textbf{0.241}} & 1.000 & 0.067\\
\textbf{Angry} & \textcolor{red}{\textbf{0.290}} & \textcolor{red}{\textbf{0.164}} & 0.067 & 1.000\\
\bottomrule
\end{tabular}
        \end{center}
	\vspace{10pt}
		\item Consequently, there is hope to achieve better performance by tackling the targets \emph{simultaneously}.

    \end{itemize}}

\end{vbframe}

\begin{frame}{Multi-target prediction: Characteristics}

	\small
		A multi-target prediction setting is characterized by instances $\xv \in \Xspace$ and targets $m \in [1, 2, \ldots, l]$ with the following properties: 

		\begin{itemize} \small
	
			\item A training data set $\D$ consists of $(\xi, \yim)$, where $\yim \in \Yspace_m$ is the label for target $m$.  
			
			\item $n$ instances and $l$ targets are observed during training. Thus, the scores $\yim$ can be arranged in an $n \times l$ matrix $Y$. Note $Y$ may have missing values.
	
			\item The label space $\Yspace_m$ can be nominal, ordinal or real-valued.  
		
			\item The goal is to predict scores for any pair $(\xv, m) \in \Xspace \times [1, 2, \ldots, l]$.  
            \vspace{10pt}
		
		\end{itemize}
% 
%
In the conventional MTP setting there is no side information for targets available. 
%	
%Note that we abuse here the notation, as we use $\Yspace$ to denote the score set. 
%
\end{frame}

\section{Special Cases of Multi-target Prediction}

\begin{frame}{Multivariate regression}
	\small
%		
		\begin{enumerate}\small
%			
			\item $|\Tspace|=l$ $\leadsto$ all targets are observed during training. 
	
			\begin{minipage}{0.45\textwidth}    
			
    			\item The score set is $\Yspace = \mathbb{R}$. 
                \vspace{10pt}
    
                \item No side information is available for targets. We can re-index the targets with $\tv = (1, 2, \ldots, l)^T$. 
            \end{minipage}
            \hfill
			\begin{minipage}{0.45\textwidth}    
				\begin{center} 	
				\includegraphics[width=0.99\textwidth,trim = 0 0 100 100,clip]{figure/Slide1} 	\tiny
				\\ Waegeman et al. (2019), Multi-target prediction:
				A unifying view on problems and methods (\href{https://arxiv.org/pdf/1809.02352.pdf}{\underline{URL}}).
  	
				\end{center}
			\end{minipage}
		\end{enumerate}
  
    \vspace{10pt}
	Example: Predict whether a protein (rows) will bind to a set of experimentally developed small molecules (columns).
%
\end{frame}



\begin{frame}{Multi-label classification}
	\small
	\begin{itemize}
		\small 

        \begin{minipage}{0.45\textwidth}  
            \item All targets are observed during training.
            \vspace{10pt}
            
            \item No side information is available for targets. 
            \vspace{10pt}
            
			\item The score matrix $Y$ has no missing values. 	
            \vspace{10pt}
			\item $\Yspace_m = \{0,1\}$ for all $m$.	
		\end{minipage}
        \hfill
		\begin{minipage}{0.45\textwidth}    
		\begin{center}
			\includegraphics[width=0.9\textwidth,trim = 0 0 100 100,clip]{figure/Slide2} \tiny
			\\ Waegeman et al. (2019), Multi-target prediction:
			A unifying view on problems and methods (\href{https://arxiv.org/pdf/1809.02352.pdf}{\underline{URL}}).
		\end{center}
		\end{minipage}
	\end{itemize}	

    \vspace{10pt}
	Example: Assign for documents (rows) the appropriate category tags (columns).

\end{frame}


\begin{frame}{Label ranking}
	\small
    In \emph{label ranking}, each instance is associated with a ranking of the targets. 

		\begin{enumerate}\small
			
			\begin{minipage}{0.45\textwidth}    
                \item All targets are observed during training. 
                \vspace{10pt}

    			\item No side information is available for targets. 		
                \vspace{10pt}
                
    			\item $Y$ has no missing values. 
                \vspace{10pt}
                
    			\item $\Yspace_m = \{1, \ldots , l\}$ for all $m$, and the scores (i.e. ranks) $\yim \neq y_{k}^{(i)}$ for all $1 \leq m,k \neq l$. 
                \vspace{10pt}
			\end{minipage}
            \hfill
			\begin{minipage}{0.45\textwidth}    
			\begin{center}
                \includegraphics[width=0.9\textwidth,trim = 0 0 100 0,clip]{figure/labelranking} \tiny
				\\ Waegeman et al. (2019), Multi-target prediction:
				A unifying view on problems and methods (\href{https://arxiv.org/pdf/1809.02352.pdf}{\underline{URL}}).
 	
			\end{center}
		\end{minipage}
		\end{enumerate}
  
    \vspace{10pt}
	Example: Predict for users (rows) their preferences over specific activities (columns).
	
\end{frame}


\begin{frame}{Multi-task learning}

		\begin{itemize}\small
			
			\begin{minipage}{0.45\textwidth}  

                \item \textbf{Not all targets are relavent for all instances}. E.g. a student may only attend one school, other labels are \textbf{irrelavent}.
                \vspace{10pt}
                
                \item All targets are observed during training.
                \vspace{10pt}

			    \item No side information is available for targets. 
                \vspace{10pt}
       
				\item Label space is homogenous across columns of $Y$, e.g., $\Yspace_m = \{0,1\}$ or $\Yspace_m = \R$ for all $m$.
                \vspace{10pt}

			\end{minipage}
            \hfill
			\begin{minipage}{0.45\textwidth}    
				\begin{center} 	
					\includegraphics[width=0.99\textwidth,trim = 0 0 100 100,clip]{figure/Slide3} \tiny
					\\ Waegeman et al. (2019), Multi-target prediction:
					A unifying view on problems and methods (\href{https://arxiv.org/pdf/1809.02352.pdf}{\underline{URL}}).	
				\end{center}
			\end{minipage}
		\end{itemize}
    \vspace{10pt}
	Example: Predict for students (rows) the final grades for a specific high-school course (columns).


\end{frame}

\begin{frame}{Remarks}

		\begin{itemize}\small
            \item It is also possible when the $m$-th task is multiclass classification. That is, $f(\xv)_m \in \R^{g_m}$ is the probability predictions for $g_m$ classes. \\
            \vspace{10pt}
            $\leadsto$ The techniques for multi-target learning are also applicalble under this setting. But the notations will be cumbersome, so we will not cover it.
            \vspace{20pt}

            \item Label space can be inhomogeneous. E.g. $\Yspace_m = \{0, 1\}$ and $\Yspace_k = \R$. \\
            \vspace{10pt}
            $\leadsto$ A mixture of multi-label classification and multivariate regression.
            
		\end{itemize}


\end{frame}


\section{Learning with Side Information on Targets}


\begin{vbframe}{Side information on targets}
    \small
    \begin{itemize}
        \item In some practical applications additional side information about the target space is available.
        \item Examples: 
        \begin{itemize} \small
            \begin{minipage}{0.45\textwidth}    
                \item Extra representation for the target molecules in drug design application (\emph{structured representation}).
            \end{minipage}
            \hfill
            \begin{minipage}{0.4\textwidth}    
                \begin{center}
                    \includegraphics[width=0.9\textwidth,trim = 0 0 50 80,clip]{figure/Slide4}	
                \end{center}
            \end{minipage}
            
            \begin{minipage}{0.45\textwidth}   
                \item Taxonomy on document categories (\emph{hierarchy}).
            \end{minipage}
            \hfill
            \begin{minipage}{0.4\textwidth}    
                \begin{center}  	
                    \includegraphics[width=0.9\textwidth,trim = 0 0 100 20,clip]{figure/Slide5}	
                \end{center}
            \end{minipage}
            
            \begin{minipage}{0.45\textwidth}    
                \item Information about schools and courses (geographical location, reputation of the school, etc.) in student mark forecasting application (\emph{feature representation}).
            \end{minipage}
            \hfill
            \begin{minipage}{0.4\textwidth}    
                \begin{center}	                                     
                    \includegraphics[width=0.9\textwidth,trim = 0 0 100 20,clip]{figure/Slide6} \tiny
                    \\ Waegeman et al. (2019), Multi-target prediction: A unifying view on problems and methods (\href{https://arxiv.org/pdf/1809.02352.pdf}{\underline{URL}}).	
                \end{center}
            \end{minipage}
        \end{itemize}
        
        \footnotesize
        
        \item Such problems are often referred to as dyadic prediction, link prediction, etc.
        
        \item Generally speaking, such settings cover problems that obey the four properties listed in the MTP definition.
        
        \item Scores $\yim$ can be arranged in a matrix $Y$, which is often sparse. 
        
        \item Thus, \emph{dyadic prediction} can be seen as \emph{multi-target prediction with target features}. 
    \end{itemize}
\end{vbframe}


\begin{frame}{Inductive vs.\ Transductive Learning}
	\footnotesize
	\begin{itemize} 
		\item In all of the previous problems, 
		\begin{enumerate} \footnotesize
			\item predictions need be be generated for novel instances, 
            \vspace{10pt}
			\item targets are known beforehand and observed during training.
		\end{enumerate}
		\item These problems are \emph{inductive} w.r.t.\ instances (1) and \emph{transductive} w.r.t.\ targets (2).
        \vspace{20pt}


		\begin{minipage}{0.45\textwidth}    
			\item 	\emph{Side information} is important for generalizing to novel targets that are unobserved during training e.g.
            \begin{itemize}
                \footnotesize
                \item a novel target molecule in the drug design,
                \item a novel tag in the document annotation,
                \item a novel school in the student grading.
            \end{itemize} 
		\end{minipage}
        \hfill
		\begin{minipage}{0.45\textwidth}    
			\begin{center}
			
				\tiny{$$\qquad g(.,.): \mbox{target similarity}$$}
				\includegraphics[width=0.9\textwidth,trim = 0 0 0 50,clip]{figure/Slide7}  \tiny
				\\ Waegeman et al. (2019), Multi-target prediction:
				A unifying view on problems and methods (\href{https://arxiv.org/pdf/1809.02352.pdf}{\underline{URL}}).
	
			\end{center}
		\end{minipage}
	\end{itemize}
\end{frame}

\begin{frame}{Subdivision of different learning settings}

	\footnotesize
	One can distinguish between four different learning settings:

	\begin{itemize} \footnotesize
	
		\item Setting A --- transductive w.r.t. targets and the instances. The goal is to predict missing values of the score matrix (\emph{matrix completion problem}).
        \vspace{10pt}

		\item Setting B --- transductive w.r.t.\ the targets and inductive w.r.t. the instances.
        \vspace{10pt}

	\end{itemize}

	\begin{minipage}{0.55\textwidth}
		\begin{itemize} \footnotesize
			\item Setting C --- inductive w.r.t.\ the targets and transductive w.r.t. the instances. $\leadsto$ Some targets are not observed during training but may appear at prediction time.
            \vspace{10pt}

			\item Setting D --- inductive w.r.t.\ both the targets and the instances (\emph{zero-shot learning}).
            \vspace{10pt}
            
		\end{itemize}
	\end{minipage}
    \hfill
	\begin{minipage}{0.4\textwidth}
	   \center
	   \includegraphics[width=0.99\textwidth,trim = 0 0 50 0,clip]{figure/Slide16}  \tiny
	   \\ Waegeman et al. (2019), Multi-target prediction: A unifying view on problems and methods (\href{https://arxiv.org/pdf/1809.02352.pdf}{\underline{URL}}).
	\end{minipage}
\end{frame}



%
\endlecture
\end{document}
